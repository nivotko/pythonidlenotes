\documentclass[14pt]{extarticle}
%\usepackage[utf8]{inputenc}
%\usepackage[english]{babel}

\usepackage{tikz}
\usepackage{geometry}
\geometry{
	a4paper,
	total={170mm,257mm},
	left=20mm,
	top=20mm,
}
\usepackage{color}
\usepackage{amsfonts}
\usepackage{amsthm}
\usepackage{amssymb}
\usepackage{amsmath}
\usepackage{mathtools}
\usepackage{bussproofs}
\newcommand{\iput}{\texttt{in}}
\newcommand{\oput}{\texttt{out}}
\newcommand{\case}[3]{\texttt{case} \, {#1} \, \texttt{of} \, {#2} \, \texttt{in} \, {#3}}
\newcommand{\diff}[2]{\texttt{diff}[{#1},{#2}]}


\title{Python Programming with IDLE}
\author{Raja Oktovin P. Damanik}
\begin{document}
\maketitle

\section{Print and Input}

\section{Data and Type}

\section{Variables and Assignment}

\section{Expression, Statement, and Using modules}

\section{Branching}

\section{Iteration}

\section{List, Set, Tuples, and Dictionary}

\section{Function}

\section{Object and Class}

\section{Exercise}

\subsection{Exercise for Section 1 to 4}
\emph{The following tasks can be done by using only Python commands mentioned in Section 1 to 4.}

\begin{enumerate}
	\item Write a program to calculate the volume of water in liters when one centimeter of water falls on one hectare.
	\item A day has 86,400 seconds (24*60*60). Given a number in the range 1-86,400, output the current time as hours, minutes,and seconds with a 24-hour clock. For example, 70,000 second is 19 hours, 26 minutes, and 40 seconds.
	\item If the lengths of the two parallel sides of a trapezoid are $X$ meters and $Y$ meters, respectively, and the heightis $H$ meters, what is the are of the trapezoid? Write Python code to output the area.
	\item Simple interest is calculated by the product of of the principal, number of years, and interest, all divided by 100. Write code to calculate he simple interest on a pricipal amount of \$10,000 for a duration of 5 years with the rate of interest equal to 12,5\%.
	\item Consider a triangle with sides of length $3,7,$ and $9$. The law of cosines states that given three sides of a triangle ($a$, $b$, and $c$) and the angle $C$ between sides $a$ and $b$; $c^2=a^2+b^2-2ab \cos (C)$. Write Python code to calculate the three angles in the triangle.
	\item (Using modules) Python comes with hunders of modules. here is a challenge for you: find a module that you can import that will generate today's date so you can print it. Use your favorite search engine for help in finding which module you need and how to use it. In the end, you task is to do the following:
	
	\texttt{>>> print("Today's date is:", X)}
	
	\texttt{Today's date is: 2009-05-22}
	
	\item Body Mass Index (BMI) is a number calculated from a person's weight and height. According to the Centers for Disease Control, tyhe BMI is a fairly reliable indicator of body fatness for most people. BMI does not measure body fat directly, but research has shown that BMI correlates to direct measures of body fat, such as underwater weighing and dual energy X-ray absorptiometry. The formula for BMI is $weight/height^2$ WHERE $weight$ is in kilograms and $heigh$ is in meters.
	\begin{enumerate}
		\item Write a program that prompts for metric weight and height and outputs the BMI.
		\item Write a program that prompts for weight in pounds and height in inches, converst the avlues to metric (kilograms and meters), and then calculate the BMI.
	\end{enumerate}
	\item Write a program taht prompts the user to input an integer that represents cents. The program will then calculate the smallest combination of coins that the user has. For example, 32 cents is 1 quarter (25 cents), 1 nickle (5 cents), and 2 pennies (1 cent).
\end{enumerate}

\subsection{Exercise for Section 5 and 6}

\begin{enumerate}
	\item In an earlier set of exercises, you were asked to calculate one's BMI. Augment that program by printing out where that BMI fits in the CDC standard wright status categories:
	\begin{center}
		\begin{tabular}{|c|c|} \hline
		BMI & Weight Status \\ \hline
		Below 18.5 & Underweight \\ \hline
		18.5-24.9 & Normal \\ \hline
		25.0-29.9 & Overweight \\ \hline
		30.0 and above & Obese \\ \hline
		\end{tabular}
	\end{center}
	\item Write a program that counts the numebr of odd numbers, even numbers, and squares from 2 to 25 (inclusive).
	\item Simplify the following code so that it does not use any \texttt{break} or \texttt{continue} statements, yet the logic is equivalent.
	
	\texttt{counter = 0}
	
	\texttt{while True:}
	
	\quad \quad \texttt{if counter >= 100:}
	
	\quad \quad \quad \quad \texttt{break}
	
	\quad \quad \texttt{elif counter \% 2 == 0:}
	
	\quad \quad \quad \texttt{counter += 1}
	
	\quad \quad \quad \quad \texttt{continue}
	
	\quad \quad \texttt{else:}
	
	\quad \quad \quad \quad \texttt{counter *= 2}
	
	\quad \quad \quad \quad \texttt{continue}
	
	\item 
	\begin{enumerate}
		\item Write a program that prompts and integer -- let's call it $X$-- and then finds the sum of $X$ consecutive integers starting at 1. That is, if $X=5$, you will find teh sum of $1+2+3+4+5=15$.
		\item Modify your program by enclosing your loop in another loop so that you can find consecutive sums. For example, if 5 is entered, youwill find five sums of consecutive numbers:
		
		\begin{tabular}{|l|c|l|}
		1 & = & 1 \\
		1+2 & = & 3 \\
		1+2+3 & = & 6 \\
		1+2+3+4 & = & 10 \\
		1+2+3+4+5 & = & 15
		\end{tabular}
		Print only each sum, not the arithmetic expression.
		\item Print also the arithmetic expression.
	\end{enumerate}
	\item Write a program that prompts for an integer and prints the integer, but if something other than that an integer is input, the program keeps asking for an integer. Here is a simple session:
	
	\texttt{Input an integer: abc}
	
	\texttt{Error: try again}
	
	\texttt{Input an integer: 4a}
	
	\texttt{Error: try again}
	
	\texttt{Input an integer: 2.5}
	
	\texttt{Error: try again}
	
	\texttt{Input an integer: 123}
	
	\texttt{The integer is: 123}
	\item A positive integer is called \emph{perfect} if the sum of all its proper divisors equals the number. For example, 6 is a perfect number since $6=1+2+3$.
	
	Write a short program that will:
	\begin{itemize}
		\item prompt the user for a number,
		\item print out whether the number is a perfect number,
		\item prompt the user for another numbe rif the input was not a perfect square.
	\end{itemize}
	\item Write a program that checks to see if a  number $N$ is a prime. A simple approach checks all numbers from 2 up to $N$, but after some point numbers are checked that need not be checked. For example, numbers greater than $\sqrt{N}$ need not be checked. Write a program that checks for primality and avoids those unnecessary checks. Remember to import the \texttt{math} module.
	\item Find the two-digit number such that when you square it, gthe resulting three-digit number has its rightmost two digits the same a sthe original two-digit number. That is, for a number in form $AB$, $AB*AB=CAB$ for some $C$.
	\item A famous puzzle follows.
	\begin{center}
		\begin{tabular}{ccccc}
			&S&E&N&D \\
		+	&M&O&R&E \\ \hline
		M	&O&N&E&Y
		\end{tabular}
	\end{center}
	What integer values can be substituted to make the addition correct? Write a program to solve this puzzle. \emph{Hint: Brute force works well --- try all possibilities.}
	\item Write a program to find 
\end{enumerate}
\end{document}